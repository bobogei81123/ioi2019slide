\documentclass[standalone]{beamer}

\begin{document}

\section{組合}

\begin{frame}{\btitle{一些基本題目}}
  \begin{enumerate}[<+->]
    \item $x_1 + x_2 + \dots + x_n = m$,且每個 $x_i \geq 0$ 的方法數:
      \onslide<+->{ \begin{flalign*} \implies {m+n-1 \choose m} && \end{flalign*} }
    \item 用 $1 \times 2$ 瓷磚(可旋轉)排滿 $2 \times n$ 長方形的方法數: \\
      \onslide<+->{$\mathtt{Fib}_n = \mathtt{Fib}_{n-1} + \mathtt{Fib}_{n-2}$, $\mathtt{Fib}_1 = 1$, $\mathtt{Fib}_0 = 0$}
  \end{enumerate}
  \onslide<+->
  \begin{missue}
  \begin{figure}
    \includegraphics[width=0.5\textwidth]{figure/fib.png}
  \end{figure}
  \end{missue}
\end{frame}

\begin{frame}{\btitle{費氏數列}}
  \[
  \begin{bmatrix}
    F_{n+1} \\
    F_{n}
  \end{bmatrix}
  =
  \begin{bmatrix}
    1 & 1 \\
    1 & 0 \\
  \end{bmatrix}
  \begin{bmatrix}
    F_{n} \\
    F_{n-1}
  \end{bmatrix}
\]
\pause

如果求的是模一個數 $m \implies$ 快速冪:$\ord(\log n)$。
\end{frame}

\begin{frame}{\btitle{一些基本題目}}
  \begin{enumerate}[<+->]
    \setcounter{enumi}{2}
  \item 把正 $n+2$ 邊型三角分割的方法數:\onslide<+->{卡特蘭數 
      \[ C_{m+1} = \sum_k C_k C_{m-k}, \ C_0 = 1 \]
    }
  \item $n!$ 種 $1$ 到 $n$ 的排列 $p$,滿足 $p_i \neq i, \, \forall i$ 的錯排數:
    \onslide<+->{\[ \sum_{k = 0}^n (-1)^k {n \choose k} (n-k)! \]}
  \end{enumerate}
\end{frame}

\begin{frame}{\btitle{一些基本題目}}
  \begin{enumerate}[<+->]
    \setcounter{enumi}{4}
  \item $n$ 個點可以構成的生成樹的個數,點視為相異的:\onslide<+->{$n^{n-2}$} \\[2ex]
  \item 給 $n$ 個數字 $A = \{a_1, a_2, \dots, a_n\}$,有多少個 $A$ 的子集 $B$
    使得 $\bigoplus_{x \in B} x = 0$,$\bigoplus$ 表示 bit \texttt{XOR}:
      \onslide<+->{\[2^{n - \operatorname{rank}(A)}\]}
  \end{enumerate}
\end{frame}

\begin{frame}{\btitle{目標}}
  \begin{problem}[經典問題]
    一個項鍊由 $N$ 個寶石串成,有 $M$ 種不同的寶石,並且因為項鍊是環形的,所以旋轉相同視為相同的。
    請問有多少種不同的項鍊?
  \end{problem}
  \pause

  \begin{enumerate}[<+->]
    \item 有很多\emph{物體}構成一個集合 $X$。
    \item 有一些\emph{旋轉} $G$。
    \item 旋轉相同視為相同,也就是旋轉會把物件劃分成許多等價類。
  \end{enumerate}
\end{frame}

\newcommand{\chain}[6][]{
  \begin{scope}[#1 shift={#2}, w/.style={fill=white}, b/.style={fill=black}, nd/.style={draw, circle}]
    \draw (0, 0) circle (0.3);
    \node[nd] at (0:0.3) [#3] {};
    \node[nd] at (90:0.3) [#4] {};
    \node[nd] at (180:0.3) [#5] {};
    \node[nd] at (270:0.3) [#6] {};
  \end{scope}
}

\begin{frame}{\btitle{Burnside}}
  \begin{columns}
    \begin{column}{0.6\textwidth}
      \begin{figure}
        \begin{tikzpicture}[x=1.3cm, y=-1.3cm]
          \chain{(0, 0)}{w}{w}{w}{w}
          \chain{(3, 0)}{b}{b}{b}{b}

          \chain{(0, 1)}{w}{w}{w}{b}
          \chain{(1, 1)}{w}{w}{b}{w}
          \chain{(2, 1)}{w}{b}{w}{w}
          \chain{(3, 1)}{b}{w}{w}{w}

          \chain{(0, 2)}{b}{w}{b}{w}
          \chain{(1, 2)}{w}{b}{w}{b}

          \chain{(0, 3)}{b}{b}{w}{w}
          \chain{(1, 3)}{w}{b}{b}{w}
          \chain{(2, 3)}{w}{w}{b}{b}
          \chain{(3, 3)}{b}{w}{w}{b}

          \chain{(0, 4)}{b}{b}{b}{w}
          \chain{(1, 4)}{b}{b}{w}{b}
          \chain{(2, 4)}{b}{w}{b}{b}
          \chain{(3, 4)}{w}{b}{b}{b}
        \end{tikzpicture}
        \caption{$M = 2$, $N = 2$ 的例子}
      \end{figure}
    \end{column}
    \vrule{}
    \begin{column}{0.4\textwidth}
      \pause
      \begin{itemize}[<+->]
        \itemsep=1ex
        \item 物體有 $\abs{X} = 2^4 = 16$ 種。
        \item 令 $a$ 為旋轉 $90^\circ$,則
          $G = \{1, a, a^2, a^3\}$。
        \item 最後答案是 $5$。
      \end{itemize}
    \end{column}
  \end{columns}
\end{frame}

\begin{frame}
  \begin{definition}
    \begin{enumerate}[<+->]
      \item $G_x \defeq \{ g \in G : g x = x \}$,也就是固定 $x$ 下,所有不會動到 $x$ 的作用。
      \item $X^g \defeq \{ x \in X : g x = x \}$,也就是固定一個作用 $g$ 下的\emph{不動點}。
      \item $G x \defeq \{ gx : g \in G \}$,也被稱作是 $x$ 在 $G$ 下的\emph{軌道}。
      \item $X / G \defeq \{ Gx : x \in X \}$,也就是 $X$ 在 $G$ 下所有的軌道。
    \end{enumerate}
  \end{definition}
\end{frame}

\begin{frame}{\btitle{目標}}
  \begin{columns}
    \begin{column}{0.6\textwidth}
      \begin{figure}
        \begin{tikzpicture}[x=1.3cm, y=-1.3cm, remember picture]
          \chain{(0, 0)}{w}{w}{w}{w}
          \chain{(3, 0)}{b}{b}{b}{b}

          \chain{(0, 1)}{w}{w}{w}{b}
          \chain{(1, 1)}{w}{w}{b}{w}
          \chain{(2, 1)}{w}{b}{w}{w}
          \chain{(3, 1)}{b}{w}{w}{w}

          \chain{(0, 2)}{b}{w}{b}{w}
          \chain{(1, 2)}{w}{b}{w}{b}
          \node (burn_x) at (1.4, 2) {};

          \chain{(0, 3)}{b}{b}{w}{w}
          \chain{(1, 3)}{w}{b}{b}{w}
          \chain{(2, 3)}{w}{w}{b}{b}
          \chain{(3, 3)}{b}{w}{w}{b}

          \chain{(0, 4)}{b}{b}{b}{w}
          \chain{(1, 4)}{b}{b}{w}{b}
          \chain{(2, 4)}{b}{w}{b}{b}
          \chain{(3, 4)}{w}{b}{b}{b}
        \end{tikzpicture}
      \end{figure}
    \end{column}
    \vrule{}
    \begin{column}{0.4\textwidth}
      \pause
      令 \tikzoverlay{burn_t}{$x$} 為箭頭所指的物體,
      $a$ 為旋轉 $90^\circ$。
      \begin{itemize}[<+->]
        \itemsep=1ex
        \item $G_x = \alt<+->{\{1, a^2\}}{\ ?}$
        \item $X^{a^2} = \ ?$
        \item $G_x = \ ?$
        \item $x^G = \ ?$
        \item $X/G = \ ?$
      \end{itemize}
      \begin{tikzpicture}[overlay, remember picture]
        \draw[-latex] ($(burn_t)+(0, 0.2)$) ..controls ($(burn_t) + (0, 1)$) and ($(burn_t) + (-1.0, 0.7)$) .. ($(burn_t) + (-1.5, 0)$)
         ..controls ($(burn_t) + (-2.0, -0.7)$) and ($(burn_x) + (3.0, 0)$) .. (burn_x);
      \end{tikzpicture}
    \end{column}
  \end{columns}
\end{frame}

\begin{frame}{\btitle{目標}}
\begin{theorem}[Burnside lemma]
  \[ \abs{X / G} = \frac{1}{\abs{G}} \sum_{g \in G} \abs{X^g} \]
\end{theorem} \pause
也就是說要算把旋轉相同視為相同的個數:\pause \disskip
\begin{enumerate}[<+->]
  \item 對於每個旋轉 $g$,計算在這個旋轉下的不動點的個數 $\abs{X^g}$。
  \item 把這些數字加起來除以 $\abs{G}$ 就是答案。
\end{enumerate}
\end{frame}

\newcommand{\hiltch}[2]{
 \fill<#1>[red, opacity=0.2] #2 circle (0.48);
 \draw[draw=none] #2 circle (0.48);
}
\begin{frame}

  \begin{columns}
    \begin{column}{0.6\textwidth}
      \begin{figure}
        \begin{tikzpicture}[x=1.3cm, y=-1.3cm, remember picture]
          \chain{(0, 0)}{w}{w}{w}{w}
          \chain{(3, 0)}{b}{b}{b}{b}
          \hiltch{2,4,6,8}{(0, 0)}
          \hiltch{2,4,6,8}{(3, 0)}

          \chain{(0, 1)}{w}{w}{w}{b}
          \chain{(1, 1)}{w}{w}{b}{w}
          \chain{(2, 1)}{w}{b}{w}{w}
          \chain{(3, 1)}{b}{w}{w}{w}
          \hiltch{2}{(0, 1)}
          \hiltch{2}{(1, 1)}
          \hiltch{2}{(2, 1)}
          \hiltch{2}{(3, 1)}

          \chain{(0, 2)}{b}{w}{b}{w}
          \chain{(1, 2)}{w}{b}{w}{b}
          \hiltch{2,6}{(0, 2)}
          \hiltch{2,6}{(1, 2)}

          \chain{(0, 3)}{b}{b}{w}{w}
          \chain{(1, 3)}{w}{b}{b}{w}
          \chain{(2, 3)}{w}{w}{b}{b}
          \chain{(3, 3)}{b}{w}{w}{b}
          \hiltch{2}{(0, 3)}
          \hiltch{2}{(1, 3)}
          \hiltch{2}{(2, 3)}
          \hiltch{2}{(3, 3)}

          \chain{(0, 4)}{b}{b}{b}{w}
          \chain{(1, 4)}{b}{b}{w}{b}
          \chain{(2, 4)}{b}{w}{b}{b}
          \chain{(3, 4)}{w}{b}{b}{b}
          \hiltch{2}{(0, 4)}
          \hiltch{2}{(1, 4)}
          \hiltch{2}{(2, 4)}
          \hiltch{2}{(3, 4)}
        \end{tikzpicture}
      \end{figure}
    \end{column}
    \vrule{}
    \begin{column}{0.4\textwidth}
      \begin{enumerate}[<+->]
        \item $\abs{X^{1}} = \onslide<+->{16} $
        \item $\abs{X^{a}} = \onslide<+->{2} $
        \item $\abs{X^{a^2}} = \onslide<+->{4} $
        \item $\abs{X^{a^3}} = \onslide<+->{2} $
      \end{enumerate}
      \pause

      \[ \frac{16 + 2 + 4 + 2}{4} = 6 \]
    \end{column}
  \end{columns}
\end{frame}

\begin{frame}
  \begin{proof}
  \begin{align*}
    \onslide<+->{\frac{1}{\abs{G}} \sum_{g \in G} \abs{X^g}} &\onslide<+->{= \frac{1}{\abs{G}}
    \sharp \big\{ (x, g) \mid x \in X, g \in G, x g = g \big\}} \\
    & \onslide<+->{= \frac{1}{\abs{G}} \sum_{x \in X} \abs{G_x}} \\
    & \onslide<+->{{=} \frac{1}{\abs{G}} \sum_{x \in X} \frac{\abs{G}}{\abs{Gx}}}\\
    & \onslide<+->{{=} \sum_{Gx \in X / G} \ \sum_{x \in Gx} \frac{1}{\abs{Gx}} = \sum_{Gx \in X/G} 1 }\\
    & \onslide<+->{= \abs{X / G}}
  \end{align*}
  \end{proof}
\end{frame}
\end{document}
