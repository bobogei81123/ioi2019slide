\documentclass[standalone]{beamer}

\begin{document}

\section{\texttt{C++} 心得分享}

\begin{frame}[fragile]{\btitle{Initializer list \& for range}}
  \begin{minted}{cpp}
int a[3] = {};
int a[3] = {0}; // in C
int b[3] = {1, 2, 3};
vector<int> C = {1, 2, 3};
pair<int, int> p = {2, 4};
  \end{minted}
  \pause

  \begin{minted}[linenos=false]{cpp}
vector<int> V = {1, 2, 3};
for (auto x: V) {}
  \end{minted}
\end{frame}

\begin{frame}[fragile]{\btitle{Some tricks}}
  \begin{minted}[linenos=false]{cpp}
    #define int long long
    
    int32_t main() {
    }
  \end{minted}
  \pause

  \begin{minted}[linenos=false]{cpp}
__int128 a;
long double b;
__float128 c;
  \end{minted}
\end{frame}

\begin{frame}[fragile]{\btitle{Some tricks}}
  \begin{minted}[linenos=false]{cpp}
    #include <ext/pb_ds/assoc_container.hpp>
    using namespace __gnu_pbds;
    typedef tree<int, null_type, less<int>,
        rb_tree_tag, tree_order_statistics_node_update> set_t;
    typedef cc_hash_table<int, int> umap_t;
  \end{minted}
  \pause

  \begin{minted}[linenos=false]{cpp}
    priority_queue<int, greater<int>> pq;
  \end{minted}
\end{frame}

\begin{frame}[fragile]{\btitle{Some tricks}}
  \begin{minted}[linenos=false]{cpp}
    int _A[MX];
    int *A = _A + MX/2; // A[-MX/2]
  \end{minted}
  \pause

  \begin{minted}[linenos=false]{cpp}
    for (int i = 0; i < (1<<n); i++) {
        // for j $\subsetneq$ i
        for (int _j = i; _j; _j = (_j-1) & i) { j = i ^ _j }
    }
  \end{minted}
  \pause

  \begin{minted}[linenos=false]{cpp}
    hypot(a, b) // sqrt(a*a + b*b); Since c++11
    clamp(x, lo, hi) // min(max(x, lo), hi) Since c++17
  \end{minted}
\end{frame}

\begin{frame}[fragile]{\btitle{Some tricks}}
  \begin{minted}[]{cpp}
    while (l < r) {
        int mid = (l+r) / 2;
        bool flag;
        if (...) {
            flag = true;
            goto loop_end;
        }
    loop_end:
        if (...) l = mid+1;
        else r = mid;
    }
  \end{minted}
\end{frame}

\begin{frame}[fragile]{\btitle{???}}
  \begin{minted}[]{cpp}
    while (l < r) {
        int mid = (l+r) / 2;
        try {
            if (...) throw true;
        } catch (bool x) {
            if (...) l = mid+1;
            else r = mid;
        }
    }
  \end{minted}
  \pause
  很慢…
\end{frame}

\begin{frame}[fragile]{\btitle{Lambda}}
  \begin{minted}[]{cpp}
    while (l < r) {
        int mid = (l+r) / 2;
        bool flag = []() {
            if (...) return true;
        };
        if (...) l = mid+1;
        else r = mid;
    }
  \end{minted}
\end{frame}

\begin{frame}[fragile]{\btitle{Lambda}}
  \begin{minted}[linenos=false]{cpp}
    sort(begin(vec), end(vec),
         [](int a, int b) { return a > b; });
  \end{minted}

  \begin{minted}[linenos=false]{cpp}
    [=x, &y] (int a, bool b) { ... };
  \end{minted}
\end{frame}

\begin{frame}[fragile]{\btitle{\texttt{C++} IO}}
  \begin{minted}[linenos=false]{cpp}
    ios_base::sync_with_stdio(0);
    cin.tie(0); // speed up

    cout << x << endl; // endl is slow!!
    #define endl '\n'
  \end{minted}

  \begin{minted}[linenos=false]{cpp}
    for (int i=0; i<vec.size(); i++)
        cout << v[i] << " \n"[i == ((int)vec.size() - 1)];
  \end{minted}
\end{frame}


%\begin{frame}[fragile]{\btitle{看過這些的表示你老了}}
  %\begin{minted}[linenos=false]{cpp}
    %#define FOR(i,c) for(__typeof((c).begin()) \
                         %i=(c).begin(); i!=(c).end(); i++)

    %vector<pair<int, int> > v;
  %\end{minted}
%\end{frame}

\begin{frame}[fragile]{\btitle{大絕招}}
  \begin{minted}[linenos=false]{cpp}
#pragma GCC optimize ("O3") 
  \end{minted}
  \pause

  其他的還有 SIMD 等等。
\end{frame}

\begin{frame}[fragile]{\btitle{Increase stack}}
  \begin{minted}{cpp}
#include <sys/resource.h>
void increase_stack_size() {
    const rlim_t ks = 64*1024*1024;
    struct rlimit rl;
    int res = getrlimit(RLIMIT_STACK, &rl);
    if (res == 0){
        if (rl.rlim_cur < ks){
            rl.rlim_cur = ks;
            res=setrlimit(RLIMIT_STACK, &rl);
        }
    }
}
  \end{minted}
\end{frame}

\begin{frame}[fragile]{\btitle{???}}
  \begin{problem}
    給你 $(x_1, y_1)$, $(x_2, y_2)$,請你計算他們的曼哈頓距離。
    輸入:一行四個 $\pm 10^9$ 內整數用空白格開。
  \end{problem} \pause \disskip
  \begin{minted}{cpp}
    #include<bits/stdc++.h>
    using namespace std;
    int x1, y1, x2, y2;

    int main() {
        cin >> x1 >> y1 >> x2 >> y2;
        cout << abs(x1 - x2) + abs(y1 - y2) << endl;
    }
  \end{minted} 
\end{frame}

\begin{frame}[fragile]{\btitle{static}}
  \begin{minted}[linenos=false]{cpp}
    static int C; // (1)
    static int f() { // (2)
        static A[30]; // (3)
    }
  \end{minted} 
  \begin{minted}[linenos=false]{bash}
    > g++ -O2 -Wall -Wextra -Wshadow -o a a.cpp
  \end{minted} 
\end{frame}

\begin{frame}[fragile]{\btitle{Undefined behavior}}
  看過很神奇的寫法: \disskip
  \begin{minted}[linenos=false]{cpp}
    struct Tree {
        Tree *l, *r;
        void walk() {
            if (this == NULL) return;
            l->walk();
            r->walk();
        }
    }
  \end{minted}
  \pause
  這是 Undefined behavior,非常不推薦!
\end{frame}

\begin{frame}[fragile]{\btitle{Undefined behavior}}
  \begin{enumerate}[<+->]
    \item Signed integer overflow
    \item Out of bound
    \item Uninitialized scalar
    \item Infinite loop without side-effects
  \end{enumerate}
\end{frame}

\begin{frame}[fragile]{\btitle{找 Bug}}
  \begin{minted}[]{cpp}
    int minus(int a, int b) {
        return a - b;
    }
    int input() {
        int t;
        cin << t;
        return t;
    }
    int main() {
        int a, b;
        cout << minus(input(), input()) << endl;
    }
  \end{minted}
\end{frame}

\begin{frame}[fragile]{\btitle{找 Bug}}
  \begin{minted}[]{cpp}
    int minus(int a, int b) {
        return a - b;
    }
    int input() {
        int t;
        cin << t;
        return t;
    }
    int main() {
        int a, b;
        cout << minus(input(), input()) << endl;
    }
  \end{minted}
\end{frame}
\begin{frame}[fragile]{\btitle{找 Bug}}
  \[ \onslide<+->{\Expect[t_\text{AC}]\ = t_\text{\tiny 寫Code}}
  \onslide<+->{ + p_\text{\tiny 出bug} \big( \Expect[t_\text{\tiny 你debug}] }
  \onslide<+-> { + p_\text{\tiny 你 de 不出來} \Expect[t_\text{\tiny 隊友debug}] \big) } \]
  \disskip
  \begin{itemize}
    \item<+-> 變數名稱取好一點。
    \item<+-> 空白多用一點。
  \end{itemize}
  \pause

    \begin{minted}[linenos=false]{cpp}
      LL mx[ 3 ] = {};
      for( size_t i = 1 ; i < f[ 1 ].size() ; i ++ )
          mx[ 1 ] = max( mx[ 1 ] , f[ 1 ][ i ] - f[ 1 ][ i - 1 ] );
    \end{minted}
\end{frame}
\end{document}
